\documentclass[]{article}
% chktex-file 21 chktex-file 46

\usepackage{amsmath}
\usepackage{mathtools}

\usepackage{xcolor}
\pagecolor{black} 
\color{white}

\usepackage[margin=20mm]{geometry}

\author{Ciaran}
\title{Defining distance units for use in HR diagram code}

\allowdisplaybreaks{}

\begin{document}
\maketitle

Throughtout this document I have referred to Carroll \& Ostlie p.61 to base my definitions from. 

One magnitude will indicate a difference of a factor \(e\) between fluxes,
\begin{equation}
    \frac{F_2}{F_1} = e^{m_1-m_2}. \label{eqn:flux ratio}
\end{equation}
Taking the logarithm (base \(e\)) we have 
\begin{equation}
    m_1-m_2 = \log \frac{F_2}{F_1}. \label{eqn:log flux ratio}
\end{equation}
We set our scale so that when $F=1$ (units), we have \(m = 0\).
Hence,
\begin{equation}
    m = - \log F . \label{eqn:log flux}
\end{equation}

Recall that we have
\begin{equation}
    F = \frac{L}{4\pi r^2}. \label{eqn:flux}
\end{equation}

Here I choose to define an absolute magnitude $M$ to be the magnitude at $1 \text{pc}$ (here we take a parsec to be an arbitrary but natural unit).
Combining equations \refeq{eqn:log flux ratio} and \refeq{eqn:flux}, we have
\begin{equation}
    e^{m-M} = \frac{F_1}{F} = {\left(\frac{d}{1\text{pc}}\right)}^2.  
\end{equation}

This gives us an equation for the distance, where \(d\) is in parsecs,
\begin{equation}
    d = e^{\frac{m-M}{2}}.
\end{equation}
Finally we compute the distance modulus,
\begin{equation}
    m-M = 2 \log d,
\end{equation}
which we can use to find that 
\begin{equation}
    M = m - 2 \log d.
\end{equation}


\end{document}