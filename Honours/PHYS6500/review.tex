% ****** Start of file apssamp.tex ******
%
%   This file is part of the APS files in the REVTeX 4.2 distribution.
%   Version 4.2a of REVTeX, December 2014
%
%   Copyright (c) 2014 The American Physical Society.
%
%   See the REVTeX 4 README file for restrictions and more information.
%
% TeX'ing this file requires that you have AMS-LaTeX 2.0 installed
% as well as the rest of the prerequisites for REVTeX 4.2
%
% See the REVTeX 4 README file
% It also requires running BibTeX. The commands are as follows:
%
%  1)  latex apssamp.tex
%  2)  bibtex apssamp
%  3)  latex apssamp.tex
%  4)  latex apssamp.tex
%
\documentclass[%
 reprint,
%superscriptaddress,
%groupedaddress,
%unsortedaddress,
%runinaddress,
%frontmatterverbose, 
%preprint,
%preprintnumbers,
%nofootinbib,
%nobibnotes,
% bibnotes,
 amsmath,amssymb,
 aps,
%pra,
%prb,
rmp,
%prstab,
%prstper,
floatfix,
]{revtex4-2}
\usepackage{macros}
\usepackage{graphicx}% Include figure files
\usepackage{dcolumn}% Align table columns on decimal point
\usepackage{bm}% bold math
\usepackage{hyperref}% add hypertext capabilities
\hypersetup{colorlinks,breaklinks,
            linkcolor=blue,urlcolor=blue,
            anchorcolor=blue,citecolor=blue}
% below allows us to // inside of an environment with hyperref
\pdfstringdefDisableCommands{%
  \def\\{}%
  \def\texttt#1{<#1>}%
}


%\usepackage[mathlines]{lineno}% Enable numbering of text and display math
%\linenumbers\relax % Commence numbering lines

% \usepackage[showframe,%Uncomment any one of the following lines to test 
% %scale=0.7, marginratio={1:1, 2:3}, ignoreall,% default settings
% %text={7in,10in},centering,
% %margin=1.5in,
% %total={6.5in,8.75in}, top=1.2in, left=0.9in, includefoot,
% %height=10in,a5paper,hmargin={3cm,0.8in},
% ]{geometry}



\begin{document}


\preprint{APS/123-QED}

\title{Extended Stellar Eclipses from Occulting Discs and Rings (catchier title needed!)}% Force line breaks with \\
% \thanks{A footnote to the article title}%

\author{Ryan M.T. White}
 \email{ryan.white@uq.edu.au}
\affiliation{%
 The University of Queensland
}%
\author{Benjamin Pope}%
\affiliation{%
 The University of Queensland
}%

\date{\today}% It is always \today, today,
             %  but any date may be explicitly specified

\begin{abstract}
    An article usually includes an abstract, a concise summary of the work
    covered at length in the main body of the article. 
    \begin{description}
        \item[Usage]
        Secondary publications and information retrieval purposes.
        \item[Structure]
        You may use the \texttt{description} environment to structure your abstract;
        use the optional argument of the \verb+\item+ command to give the category of each item. 
    \end{description}
\end{abstract}

% \keywords{Suggested keywords}%Use showkeys class option if keyword
                              %display desired
\maketitle

%\tableofcontents

\section{Introduction} \label{sec:intro}
Discs are ubiquitous throughout astrophysics, found everywhere from planetary ring systems to protostellar discs to active galactic nuclei and galaxies as a whole. All stars are born embedded within discs as a result of angular momentum conservation following the collapse of molecular clouds, with planets often forming in lockstep from these natal discs. For close binary star systems in particular, circumstellar and accretion discs are expected to form at various times as the component stars evolve and transfer mass, whether on/after the main sequence or as remnants or some combination thereof. Similarly, discs are present throughout various stages of some massive planets' lifetimes both during their formation and in the form of rings sculpted by exo-moons and their collisions or tidal breakup.

The mostly widely used method of studying stars and planets alike is in time-series photometry, where the apparent brightness of an object or system over time is called a `light curve'. Photometry is observationally cheap and can be done with a series of short exposures viewing up to thousands of targets simultaneously. In surveying the sky this way, we can see in real-time how the stars change either due to intrinsic or extrinsic factors. In the former case, stellar variability such as flares or pulsation modes are common. In the latter case, one possible phenomenon is the transit of an astrophysical companion (another star or planet) over the face of the star which reveals itself in a sharp dip in observed brightness. 

The observational characteristics of light curve eclipses, such as the dip depth and breadth, provide strong constraints on the type of eclipsing object. On account of their size, occulting stars produce a correspondingly broader and deeper transit than a planet would on the same orbit. Over the course of an orbital period, secondary eclipses are also observed where the primary star occludes the companion resulting in another dip that is shallower. When combining information of the primary and secondary eclipses together, some degeneracies or unknowns in the object parameters can be broken. In particular, spectra taken during each eclipse can be compared to that out of eclipse to determine the chemical signature and temperature of each component.

With the operation of several wide field survey programs, such as the \textit{Kepler} satellite, Zwicky Transient Facility (ZTF) and the Transiting Exoplanet Survey Satellite (TESS) among others, thousands of companion objects have been discovered with light curves. Considering that the orientation of the companion orbital plane needs to be almost exactly along the line-of-sight, relatively few companions out of the total population are detected. 

Recently, there have been an increasing number of eclipses which cannot be simply explained by a transiting planet or star. These eclipses display especially broad and deep profiles and have been well modelled as transiting dusty discs or planetary ring systems. Given that eclipsing binaries/transiting planets are already a chance observation, the fact that only a handful of these extended eclipses have been discovered establishes them as truly rare and unique phenomena.

This review is structured as...

\section{Disc Eclipsing Systems} \label{sec:disc-eclipse}
There have been several disc eclipsing binary systems observed to date. The underlying nature of these systems is not homogeneous, with a categorical divide between post- and pre-main sequence stages of stellar evolution resulting in disc formation. Historically the post-main sequence systems have been the most rigorously studied on account of their brightness making a comparatively larger population visible, although discoveries of pre-main sequence disc eclipsers have been increasingly common in recent years.  

In this section we review the broad physical and observational characteristics of the circumstellar disc eclipsing systems. One particular system stands out among these as by far the most well studied and whose nature was mysterious for over 150 years, and so we begin by discussing the prototype disc eclipser Epsilon Aurigae.


\subsection{Epsilon Aurigae} \label{sec:eps-aur}

As the prototype disc eclipsing system, Epsilon Aurigae (Eps Aur) is a binary system consisting of a F0Ia and B5V star \citep{Hoard2010ApJ}. The first indication of variability in Eps Aur was observed in 1821 by German astronomer Johann Heinrich Fritsch, when the B star enshrouded by a circumstellar dust disc eclipsed the F star resulting in a $\sim 0.8$ magnitude dip in brightness.  
While many eclipsing binary systems are known and well understood, Eps Aur stands out in particular due to its especially prolonged eclipse profile. Most eclipsing binaries have transit periods of order days or less, though the periodic occultations of Eps Aur last $\sim 700$ days and repeat every 27.1 years corresponding to the systems orbital period \citep{Stencel2011AJ}.

\begin{figure}
    \centering
    \includegraphics[width=\columnwidth]{Images/Eps-Aur-Image.png}
    \includegraphics[width=\columnwidth]{Images/Eps-Aur-Eclipse.png}
    \includegraphics[width=\columnwidth]{Images/Eps-Aur-Spectrum.png}
    \caption{As the best studied disc eclipsing system to date, there is a wealth of photometric, spectroscopic and interferometric data on Epsilon Aurigae. \emph{Top:} using the CHARA interferometric array, \citet{Kloppenborg2015ApJS} was able to reconstruct a semi-complete image of the occluding disc in Eps Aur as it transited the F star. This image is the combination of multiple epochs of reconstructions, showing the opaque disc structure moving across the face of the giant star. \emph{Middle:} the 2009-2011 eclipse of Eps Aur shows a distinct dip in brightness over $\sim 700$ days, with several bumps mid-eclipse \citep{Stencel2011AJ}. \emph{Bottom:} The spectrum of Eps Aur close to eclipse is well fit by the sum of B5 and F0 spectra together with a cool $\sim 550$K blackbody \citep{Hoard2010ApJ}.}
    \label{fig:eps-aur}
\end{figure}

The first physically motivated interpretation of this eclipse profile was made by \citet{Kuiper1937ApJ}, suggesting that the transiter is a semi-transparent and large star. Prior to the next observed transit, \citet{Kopal1954Obs} proposed a ring of material around the secondary component to explain the eclipse profile; this idea was first modelled by \citet{Huang1965ApJ} with a geometrically and optically thick disc in orbit of the F0 giant. Naturally, an extended and optically thick disc of material obscures the bright star for longer than even a giant star might. This model was refined by \citet{Wilson1971ApJ} to an inclined and thick disc of material surrounding the binary stellar companion being responsible for the transit. This inclined thin disc model featured a cavity around the central star within the disc (as would be expected based on models of accretion and dust discs) motivated by the presence of a mid-eclipse increase in brightness of the system (see the middle panel of Fig.~\ref{fig:eps-aur}, approx day 55400) which is not reproduced from a simple edge-on disc model. This model was favoured for decades, with various refinements to match new incoming photometric and spectroscopic data. Notably, \citet{Carroll1991ApJ} suggested that the disc may be thin with a variable opacity throughout the disc profile. For a detailed review of the attempts to model Eps Aur in light of observations up to 2002, see \citet{Guinan2002ASPC}.

In the time since, infrared astronomy and interferometry in particular has significantly matured. Using the CHARA array, \citet{Kloppenborg2010Natur} directly observed a section of the transiting disc at two epochs during the ingress of the 2009-2011 Eps Aur eclipse. After more observations, effectively the complete disc silhouette was imaged and reconstructed \citep[see top panel in Fig.~\ref{fig:eps-aur}]{Kloppenborg2015ApJS} which allowed the authors to select the most likely disc model. This best fitting model is of a slightly tilted disc with a decaying density towards the disc edge, in support of a disc with variable opacity. 

Despite the success in explaining some mid-eclipse variability with a variable opacity disc, the prominent mid-eclipse brightening event observed over all previous transits is not so easily modelled. For this, \citet{Budaj2011A&A} modelled the effect of dust grain size and angle dependence of the disc with a radiative transfer code, ultimately suggesting that there exists a larger, optically thin disc of material surrounding the optically thick disc. During transit, this outer disc would scatter light and the strength of this scattering has been used to constrain the inclination of the disc \citep{Muthumariappan2012MNRAS}.

Aside from physical structure, the disc has been found to have an inhomogeneous thermal profile. Modelling the system spectrum during eclipse shows a clear dependence on a 550K blackbody to explain the observed mid-infrared flux \citep[][bottom panel of Fig.~\ref{fig:eps-aur}]{Backman1984ApJ, Hoard2010ApJ}, a spectral signature attributed to the eclipsing disc. Observing during the transit means that only the region of the disc most distant from the F star (and with significant intermediate material) is visible. \citep{Backman1984ApJ} reasoned that the side of the disc facing the star should be of order $\sim 1100$K as calculated from the equilibrium temperature of material 30AU from the F0 star. Indeed, \citet{Hoard2012ApJ} showed that the spectrum of the disc viewed at opposition has clear signatures of a 1150K blackbody, in good agreement with the disc being heated by the F star \citep{Takeuchi2011PASJ, Pearson2015ApJ}. 

Since its discovery, the exact formation mechanism of the secondary's circumstellar disc has been elusive. In particular, uncertainty in distance estimates to the system has fueled longstanding debate as to whether the system is explained by a `low mass' or `high mass' scenario \citep{Lissauer1996ApJ}. In the former (corresponding to the system being closer), the F0 star would have had a ZAMS mass of $\gtrsim 7M_\odot$ which has since shed mass both through outbursts/winds and through mass transfer to the secondary, now classified as a post-AGB giant. In the latter scenario (for a larger distance), the F0 component is a $15-20M_\odot$ supergiant star near the end its evolution while the B5 star would be of order $\sim 14M_\odot$ \citep{Lambert1986PASP, Hoard2010ApJ}. Both through spectral fitting \citep{Hoard2010ApJ, Hoard2012ApJ} and evolutionary modelling \citep{Gibson2018MNRASa}, the low mass scenario has been recently preferred. An independent finding supporting this was in the discovery of a mass transfer stream between the F0 star and the disc, only visible through spectroscopy at a very narrow window in phase after periastron passage \citep{Griffin2013PASP, Gibson2018MNRASb}. The best-fit orbit yields an eccentricity of the binary of $e = 0.227$ \citep{Stefanik2010AJ}, which may be the underlying cause of the strong phase sensitivity of the mass stream. As the low mass scenario requires mass transfer between the components, this strongly disfavours the higher mass scenario where the binary stellar evolution is largely non-interacting. Further, \citet{Wolk2010AJ} published a non-detection of X-rays associated with Eps Aur effectively ruling out a model in which the disc enshrouds a compact remnant. 

Although Epsilon Aurigae is undoubtedly the best studied disc eclipsing system, much is not yet understood. There is a gap in the literature of modelling this system from first principles, e.g. with a full hydrodynamical treatment of the mass transfer and dust physics. A rigorous modelling of the disc structure, including any inhomogeneities or warping has not yet been published. This together with the poorly constrained evolutionary history of the system remain the biggest unanswered questions of Eps Aur.

\subsection{Post-Main Sequence Systems} \label{sec:postMS-disc}

In the time since the discovery of Epsilon Aurigae, and especially in the last 10 years, there has been several more disc-eclipsers found. Observing their photometry in isolation makes it difficult to determine whether a disc-eclipsing system is pre- or post-main sequence, although many of these systems have associated spectroscopic data to aid in characterisation. Those systems found with a post-main sequence component are particularly interesting from the perspective of evolved binary interactions and as gravitational wave progenitors. Along those lines, the majority of these evolved disc eclipsers fall, again, into two distinct categories: those of short ($\ll 1$ year) orbital period resembling stars such as Beta Lyrae, and those of long periods more resembling Epsilon Aurigae.

\subsubsection{Short- and Intermediate-Period Systems}
Evolved binary stars that orbit with short ($P \lesssim 20$ days) and intermediate periods ($20 \lesssim P \lesssim 100$ days) are of particular interest from a coevolving and gravitational wave/supernova perspective \citep{Sana2012Sci}. Systems containing accreting white dwarfs are expected to eventually undergo a supernova explosion, such as CI Aql in $\sim 10$Myr \citep{Lederle2003A&A, Sahman2013MNRAS}, and closely orbiting massive stars result in (any combination of) neutron star/black hole mergers within the Hubble time \citep{Kruckow2018MNRAS}. The slightly-less-evolved counterparts of these systems and other stellar remnant binaries are therefore subject to rigorous studies in an attempt to better understand their evolutionary histories.

As perhaps the most famous example of a short period binary, the Algol system is the subject of paradoxical characteristics; the more evolved K2IV star has a significantly lower mass than the main sequence B8V companion \citep{Baron2012ApJ}. The solution to this paradox lies in mass-transfer between the components, where the initially higher mass giant evolved into its Roche lobe and subsequently donated mass to the once lower mass star thereby switching the mass hierarchy. The extremely short $\sim 2.9$ day period of the system presents an impressive eclipsing binary light curve with regular and deep transits, and the binary has been directly imaged showing the geometric distortion of the donor star \citep{Baron2012ApJ}. 

It is common for persistent accretion discs to exist in Algol-like systems as the donor transfers mass through Roche lobe overflow (RLOF). The $\sim 13$ day orbital period binary Beta Lyrae is one such system, where a thick accretion disc obscures the `gainer' component at all times \citep[][for direct images]{Ak2007A&A, Zhao2008ApJ}. The Beta Lyrae accretion disc has been well modelled using photometric data of the transit profiles, but also with spectroscopic and interferometric data which allows inference of the disc temperature profile (among other parameter profiles) as well as the presence of jets \citep{Broz2021A&A}. This type of modelling \citep[together with the system V4142 Sgr, for example;][and references on other systems therein]{Rosales2023A&A} unambiguously show that the temperatures of Beta-Lyrae type accretion discs are too high to host dust formation, apparently unlike long period systems such as Eps Aur. This is corroborated by a distinct double peaked hydrogen-alpha emission in these systems, indicative of a rotating accretion disc; these features are routinely modelled to understand disc dynamics and temperature \citep[for examples modelling AU Mon]{Desmet2010MNRAS, Atwood-Stone2012ApJ}.

The key difference that the accretion disc imposes on the light curve is a more extended, smooth transit across the primary eclipse. For such close systems, the donor filling its Roche lobe smooths the secondary eclipse as a result of geometric effects \citep{Huang1963ApJ, Wilson2018ApJ}. Since the stars are so close together, reflection effects are essential to include in light curve models although the accretion disc can influence the effect of this depending on its opacity and geometry \citep{Wilson1990ApJ, Pavlovski2006A&A}. These close binary systems also have largely circularised orbits \citep[][mostly due to tidal effects]{Moe2017ApJS, Zasche2018A&A} which result in half-phase separated primary and secondary eclipses \citep[although there has been at least one discovery with a non-zero eccentricity of $e \simeq 0.021$ in][]{Miller2007ApJ}.

Opening the parameter space to so called `intermediate' period binaries (of order $20-100$ days) introduces more exotic systems than is often seen in the short period cases. One particular system is TT Nor ($P=37.35$ days) which has a clearly eccentric orbit as a result of its unequally spaced light curve eclipses \citep{Rowan2023MNRAS}. Together with the simultaneously reported light curve of star OGLE-LMC-ECL28271 ($P=27.11$ days), these transits display eclipse profiles largely unlike the typical thick, opaque discs that are typically modelled \citep{Rowan2023MNRAS} and hint at disc physics more in line with Eps Aur.  

The idea of system with a non-circular orbit poses interesting questions about the stability of the mass-transfer accretion discs in short-period systems. \citet{Mennickent2021A&A} reported a finding of a variable transit profile in the system OGLE-BLG-ECL-157529 on a timescale of 18 years, despite an associated orbital period of 24.8 days. The authors attribute the variability over successive transits to a changing disc thickness associated with inconsistent mass transfer rates. An appreciable eccentricity is expected to result in highly phase-dependent mass transfer events, which should affect the binary evolution substantially \citep{Davis2013A&A}, even so far as accelerating the overall mass-transfer. 

While RLOF occurs, orbital angular momentum conservation dictates that the orbital separation (and hence the period) must increase between the stars \citep{Zhou2018ApJ}. This has been used to explain presently large orbital periods in what are understood to be post-Algol systems (i.e. where mass transfer is no longer appreciably taking place). This raises the question, however, of how accretion discs remain in systems that are too far separated for RLOF to occur, such as in MWC 882 \citep{Zhou2018ApJ}. This remains a topic of active research.


\subsubsection{Long-Period Systems}
The mystery of the persisting lifetimes of circumstellar discs in post-main sequence systems is further complicated for those systems of orbital periods $P \gtrsim 100$ days. Even for the largest giant stars, the binary stars are well detached and no classical RLOF occurs. 


ELHC 10 -- \citep{Garrido2016MNRAS} \\
Eta Gem -- \citep{Torres2022MNRAS} \\
EE Cephei -- \citep{Mikolajewski1999MNRAS, Pienkowski2020A&A} \\
OGLE-BLG182.1.162852 -- \citep{Rattenbury2015MNRAS} \\
OGLE-LMC-ECL-11893 -- \citep{Dong2014ApJ, Scott2014ApJ} \\
OGLE-LMC-ECL-17782 -- \citep{Graczyk2011AcA} \\
Long period occultation in the nucleus of planetary nebula M 2-29 -- \citep{Hajduk2008A&A} \\
R Aquarii, a WD+M giant -- \citep{Hinkle2022ApJ} \\
V Hya -- \citep{Knapp1999A&A} \\
KIC 5273762 -- \citep{Jayasinghe2018RNAAS} \\
TYC 2505-672-1 -- \citep{Rodriguez2016AJ, Lipunov2016A&A} \\
VVV-WIT-08 -- \citep{Smith2021MNRAS} \\
Gaia17bpp -- \citep{Tzanidakis2023ApJ} \\
ASASSN-21co -- \citep{Rowan2021RNAAS} \\
OGLE-LMC-T2CEP-211 -- \citep{Pilecki2018ApJ} \\
W Crucis \citep{Pavlovski2006A&A} \\
Evolving disc profile -- \citep{Bernhard2024arXiv} \\


\subsection{Pre-Main Sequence Systems} \label{sec:preMS-disc}
Gaia21bcv -- \citep{Hodapp2024AJ} \\
V928 Tau -- \citep{van-Dam2020AJ} \\
V773 Tau -- probably pre-main-sequence \citep{Kenworthy2022A&A} \\
HMW 15 -- \citep{Cohen2003ApJ} \\
Also an evolving disc profile in KH 15D, although this is from a circumbinary disc around a T Tauri binary
 -- \citep{Hamilton2001ApJ, Hamilton2005AJ, Winn2006ApJ} \\
EPIC 220208795, a K dwarf not necessarily young -- \citep{van-der-Kamp2022A&A}

Dusty disc around minor body, $\sim 3M_J$, eclipsing system around young M dwarf -- \citep{Rappaport2019MNRAS}


\section{Dipper Stars} \label{sec:dippers}
KIC 8462852 -- \citep{Boyajian2016MNRAS} \\
HD 166191 -- \citep{Su2022ApJ} \\
ASASSN-21qj -- \citep{Marshall2023ApJ} \\
\citep{Ansdell2016ApJ}\\
\citep{Melis2021ApJ} -- dusty dipper system, potentially from planetary collision


\section{Ring Eclipsing Systems} \label{sec:ring-eclipse}
V1400 Centauri -- \citep{Mamajek2012AJ, van-Werkhoven2014MNRAS, Kenworthy2015ApJ} \\
ASASSN-21js -- \citep{Pramono2024A&A} \\
PDS 110 -- \citep{Osborn2017MNRAS} \\
Rings can be inferred from transmission spectra \citep{Ohno2022ApJ} -- need to better understand ring compositions





\section{Summary and Outlook}



\section{Outline}
\begin{itemize}
    \item Introduce light curve transits as an observationally cheap avenue for discovering binary companions and planets. Some of these transits are unusually prolonged (many times longer than a typical star/planet transit) and hint at different transient phenomena. 
    \item Finally talk about disc-eclipsing systems. 
    \begin{itemize}
        \item Disc-eclipsing binaries are seen exclusively around either very young systems, or systems with at least one evolved star. What physics gives rise to discs in each case?
        \item As before, what do we expect the transits to look like? Spectra?
        \item Describe the systems found so far. What are the evolutionary histories of the stars in these systems? How will the binary evolution affect their end products. Gravitational wave progenitors?
    \end{itemize}
    \item Introduce (briefly) dipper stars. Don't want to go into too much detail here as they're not the focus of the review (should I even mention them?)    
    \begin{itemize}
        \item Typically we expect these around very young stars. Dusty discs surrounding the stars could be protoplanetary discs with forming planetesimals, or even potentially debris clouds from planetary collisions. 
    \end{itemize}
    \item Section on ring eclipsing systems. 
    \begin{itemize}
        \item We know that gas giants in the solar system have rings, so we should expect them around other stars. 
        \item What do we expect ring transits to look like? Mention that many ring systems have multiple rings having different densities (and hence stronger dips), often with gaps in the rings, we get back-illumination from the star on the rings making the very beginning and end of transit appear brighter than the baseline. \item Also mention transmission spectroscopy?
        \item What ring systems have been seen so far? Give examples. What observational biases are there? What do these observed systems imply about the greater (unseen/invisible) population? 
    \end{itemize}
    \item Conclusions.
    \begin{itemize}
        \item What are our blindspots in finding these systems?
        \item Expect to discover these at scale with LSST, and there are likely many hiding in archival data of TESS/ZTF. 
    \end{itemize}
\end{itemize}




\begin{acknowledgments}
h
\end{acknowledgments}

% \appendix

% \section{Appendixes}

% The \nocite command causes all entries in a bibliography to be printed out
% whether or not they are actually referenced in the text. This is appropriate
% for the sample file to show the different styles of references, but authors
% most likely will not want to use it.
% \nocite{*}

% \bibliographystyle{apsrmp4-2}
\bibliographystyle{mnras} % want to use the mnras style for now so that we have links to ads
\bibliography{references}% Produces the bibliography via BibTeX.

\end{document}
%
% ****** End of file apssamp.tex ******
